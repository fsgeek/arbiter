\section{Appendix A: Scourer Convergence Data}

\subsection{Claude Code v2.1.50}

\begin{table*}[t]
\small
\begin{tabular}{@{}lllll@{}}
\toprule
Pass & Model & New Findings & Cumulative & Continue? \\
\midrule
1 & Claude Opus 4.6 & 21 & 21 & yes \\
2 & Gemini 2.0 Flash & 9 & 30 & yes \\
3 & Kimi K2.5 & 14 & 44 & yes \\
4 & DeepSeek V3.2 & 12 & 56 & yes \\
5 & Grok 4.1 & 10 & 66 & yes \\
6 & Llama 4 Maverick & 5 & 71 & yes \\
7 & MiniMax M2.5 & 20 & 91 & yes \\
8 & Qwen3-235B & 3 & 94 & no \\
9 & GLM 4.7 & 14 & 108 & no \\
10 & GPT-OSS 120B & 8 & 116 & no \\
\bottomrule
\end{tabular}
\end{table*}

\subsection{Codex CLI (GPT-5.2)}

\begin{table*}[t]
\small
\begin{tabular}{@{}lllll@{}}
\toprule
Pass & Model & New Findings & Cumulative & Continue? \\
\midrule
1 & DeepSeek V3.2 & 10 & 10 & yes \\
2 & Grok 4.1 & 5 & 15 & no \\
\bottomrule
\end{tabular}
\end{table*}

\subsection{Gemini CLI}

\begin{table*}[t]
\small
\begin{tabular}{@{}lllll@{}}
\toprule
Pass & Model & New Findings & Cumulative & Continue? \\
\midrule
1 & DeepSeek V3.2 & 12 & 12 & yes \\
2 & Qwen3-235B & 5 & 17 & yes \\
3 & GLM 4.7 & 4 & 21 & no \\
\bottomrule
\end{tabular}
\end{table*}

\section{Appendix B: Scourer Prompt Templates}

\subsection{First Pass}

\begin{verbatim}
You are exploring a system prompt. Not auditing it, not checking it
against rules — just reading it carefully and noting what you find
interesting.

"Interesting" is deliberately vague. Trust your judgment. You might notice:
- Instructions that seem to contradict each other
- Rules that are stated multiple times in different places
- Implicit assumptions that aren't declared
- Surprising structural choices
- Scope ambiguities (when does a rule apply?)
- Things that would confuse a model trying to follow all instructions
  simultaneously
- Interactions between distant parts of the prompt
- Anything else that catches your attention

[...] After documenting what you found, document what you DIDN'T explore.
What areas did you skim? What questions occurred to you that you didn't
pursue?

Finally: should we send another explorer after you? Would another pass,
armed with your map, find things you missed?
\end{verbatim}

\subsection{Subsequent Passes}

\begin{verbatim}
You are exploring a system prompt. Previous explorers have already been
through it and left you their map. Your job is to go where they didn't.

DO NOT repeat their findings. They found what they found. You are looking
for what they missed, what they flagged as unexplored, and anything their
framing caused them to overlook.

[Previous findings and unexplored territory injected here]

Be honest about diminishing returns. Set should_send_another to FALSE if:
- Most of your findings are refinements or restatements of existing ones
- The unexplored territory is mostly about runtime behavior
- You found fewer than 3 genuinely new findings
- The prior passes have already covered the major structural, security,
  operational, and semantic categories

It is better to say "enough" than to pad findings.
\end{verbatim}

\section{Appendix C: Directed Analysis --- Claude Code Interference Patterns}

21 hand-labeled interference patterns, with severity on an impact scale:

\begin{table*}[t]
\small
\begin{tabular}{@{}lllll@{}}
\toprule
\# & Type & Blocks & Severity & Static? \\
\midrule
1 & Direct contradiction & TodoWrite mandate ↔ Commit restriction & Critical & Yes \\
2 & Direct contradiction & TodoWrite reinforcement ↔ Commit restriction & Critical & Yes \\
3 & Direct contradiction & TodoWrite mandate ↔ PR restriction & Critical & Yes \\
4 & Direct contradiction & TodoWrite reinforcement ↔ PR restriction & Critical & Yes \\
5 & Scope overlap & TodoWrite mandate ↔ TodoWrite tool (exceptions) & Major & Yes \\
6 & Priority ambiguity & Security policy ↔ Security policy (duplicate) & Minor & Yes \\
7 & Scope overlap & Conciseness ↔ TodoWrite (overhead) & Major & No \\
8 & Scope overlap & Conciseness ↔ WebSearch (sources requirement) & Minor & Yes \\
9 & Scope overlap & Task tool search ↔ Explore agent guidance & Major & Yes \\
10 & Scope overlap & Read-before-edit (general) ↔ Edit tool & Minor & Yes \\
11 & Scope overlap & Read-before-edit (general) ↔ Write tool & Minor & Yes \\
12 & Scope overlap & No-new-files (tone) ↔ Write tool & Minor & Yes \\
13 & Scope overlap & No-new-files (tone) ↔ Edit tool & Minor & Yes \\
14 & Scope overlap & Dedicated tools (policy) ↔ Bash tool & Minor & Yes \\
15 & Scope overlap & Dedicated tools (policy) ↔ Grep tool & Minor & Yes \\
16 & Scope overlap & No time estimates ↔ Asking questions & Minor & Yes \\
17 & Implicit dependency & Commit restrictions ↔ General Bash policy & Minor & Yes \\
18 & Implicit dependency & Plan mode tool restrictions ↔ Tool policy & Minor & Yes \\
19 & Scope overlap & No-emoji (tone) ↔ Edit tool & Minor & Yes \\
20 & Scope overlap & No-emoji (tone) ↔ Write tool & Minor & Yes \\
21 & Priority ambiguity & Parallel calls ↔ Commit workflow ordering & Minor & Yes \\
\bottomrule
\end{tabular}
\end{table*}

\section{Appendix D: Cost Breakdown}

Costs are from OpenRouter billing records (activity export 2026-02-27).
Claude Opus 4.6 was billed via Anthropic subscription and is excluded
from the total.

\begin{table*}[t]
\small
\begin{tabular}{@{}lll@{}}
\toprule
Model & Calls & Actual Total \\
\midrule
Kimi K2.5 & 2 & \$0.054 \\
DeepSeek R1 & 1 & \$0.054 \\
Qwen3-235B & 3 & \$0.053 \\
GLM 4.7 & 2 & \$0.039 \\
Grok 4.1 Fast & 5 & \$0.016 \\
Llama 4 Maverick & 3 & \$0.015 \\
DeepSeek V3.2 & 3 & \$0.012 \\
MiniMax M2.5 & 1 & \$0.012 \\
Gemini 2.0 Flash & 2 & \$0.005 \\
GPT-OSS 120B & 4 & \$0.003 \\
\textbf{Total} & \textbf{26} & \textbf{\$0.263} \\
\bottomrule
\end{tabular}
\end{table*}

Notes: Call count exceeds pass count (15) due to retries (Kimi K2.5
output length limit), intermediate experiments (DeepSeek R1), and the
growing prompt size as accumulated findings are passed forward. The
Qwen3-235B total includes one anomalous 175K-token prompt (\$0.046
alone), likely a routing artifact. Cost per finding: \$0.002.
