\section{Discussion}

\subsection{Prompts as Software Artifacts}

The analogy between system prompts and software systems is not
metaphorical. The same structural taxonomy (monolith, modular, flat)
applies to both, and it predicts the same classes of failure:

\begin{itemize}
\tightlist
\item
  Monoliths accumulate contradictions at internal subsystem boundaries.
\item
  Modular systems produce bugs at composition seams.
\item
  Flat systems trade capability for consistency.
\end{itemize}

This is not a coincidence. It is Conway's Law applied to a new medium:
the structure of the prompt reflects the structure of the team that
produced it. Claude Code's monolithic prompt --- with its TodoWrite
subsystem contradicting its commit workflow --- reads like the output of
separate teams adding capabilities to a shared artifact without
cross-team integration testing.

The implication is that system prompts need the same engineering
infrastructure that conventional software has: linters for internal
consistency, tests for behavioral contracts, CI/CD pipelines that catch
regressions when one section is edited. Arbiter's directed evaluation
rules are a prototype of this infrastructure.

\subsection{The Observer's Paradox}

The executing LLM smooths over contradictions via ``judgment.'' When
Claude Code encounters ``ALWAYS use TodoWrite'' and ``NEVER use
TodoWrite'' in the same context, it picks one. The system works --- most
of the time. This is precisely why the contradictions persist: they
cause no visible errors, only invisible inconsistency.

An external evaluator with formal rules surfaces what execution hides.
This is directly analogous to static analysis catching bugs that unit
tests miss: the code ``works'' in the sense that tests pass, but the
analyzer identifies structural issues that will eventually manifest as
runtime failures under the right conditions.

The thesis --- \textbf{the agent that resolves the conflict cannot be
the agent that detects it} --- has a precise formulation: the heuristic
that enables an LLM to navigate contradictory instructions is the same
heuristic that prevents it from recognizing those instructions as
contradictory. Detection requires a different vantage point.

\subsection{Severity and Epistemic Confidence}

The directed and undirected phases use different severity scales, and
this difference is informative.

\textbf{Directed analysis} uses an impact scale: \emph{critical}
(structurally guaranteed to cause incorrect behavior), \emph{major}
(will cause problems under specific conditions), \emph{minor}
(maintenance concern). This is appropriate for known interference
patterns with clearly defined failure modes.

\textbf{Undirected scouring} uses an epistemic confidence scale:
\emph{curious} (pattern noticed, no judgment on impact), \emph{notable}
(worth investigating), \emph{concerning} (likely problematic),
\emph{alarming} (structurally guaranteed to fail). This is appropriate
for discoveries where the impact depends on runtime behavior that static
analysis cannot determine.

These scales are orthogonal. A finding can be epistemically ``curious''
but operationally critical, or epistemically ``alarming'' but
operationally irrelevant. The multi-model scouring provides epistemic
provenance --- which model found which finding --- enabling downstream
consumers to weight findings by the reliability of their source.

\subsection{Cost}

The total cost of analyzing all three vendor prompts was \$0.27 USD,
verified against OpenRouter billing records:

\begin{table*}[t]
\small
\begin{tabular}{@{}lll@{}}
\toprule
Vendor & Passes & Actual Cost \\
\midrule
Claude Code & 10 & \$0.236 \\
Codex CLI & 2 & \$0.012 \\
Gemini CLI & 3 & \$0.014 \\
\textbf{Total} & \textbf{15} & \textbf{\$0.263} \\
\bottomrule
\end{tabular}
\end{table*}

Claude Opus 4.6 (pass 1, Claude Code) was billed via Anthropic
subscription and is excluded. Including a pro-rated subscription
allocation would not materially change the total.

The largest cost drivers were Kimi K2.5 (\$0.054, including one retry
that hit an output length limit), Qwen3-235B (\$0.053, including one
anomalous 175K-token prompt), and GLM 4.7 (\$0.039). GPT-OSS 120B was
the cheapest model at \$0.003 total across four calls. Initial per-model
estimates underestimated total cost by 3.8× because they assumed one API
call per model rather than accounting for retries, growing prompt sizes
as accumulated findings are passed to subsequent models, and reasoning
token overhead.

The cost per finding across all three analyses is \$0.002. At US federal
minimum wage (\$7.25/hour), the entire cross-vendor campaign costs less
than three minutes of human labor. This is not a methodological footnote
--- it is a result. It means that system prompt analysis at this level
of thoroughness is accessible to any developer with API access, not just
teams with dedicated security budgets.

\subsection{Limitations}

\textbf{Static analysis only.} This work analyzes system prompt text,
not runtime behavior. A contradiction in the prompt may or may not
manifest as incorrect behavior, depending on how the executing LLM
resolves it. We identify the structural conditions for failure, not the
failures themselves.

\textbf{Scourer findings are LLM-generated.} The undirected scourer asks
LLMs to analyze prompts, and LLMs can fabricate observations. We
mitigate this through multi-model execution (fabrications from one model
are unlikely to be independently confirmed by another) and through human
review of all findings. However, we cannot guarantee that every scourer
finding corresponds to a real problem.

\textbf{Directed analysis limited to one vendor.} The 56-block
decomposition and 21-pattern ground truth exist only for Claude Code.
Extending directed analysis to Codex and Gemini CLI requires additional
manual labeling effort.

\textbf{Gemini CLI prompt is composed.} Unlike Claude Code and Codex
CLI, whose prompts were captured as-is, Gemini CLI's prompt was composed
from TypeScript render functions with all feature flags enabled. The
actual runtime prompt depends on which features are active and may be
shorter or differently composed.

\textbf{Cost data excludes subscription allocation.} API costs are
verified against OpenRouter billing records, but Claude Opus 4.6 (one
pass) was billed via Anthropic subscription. The reported \$0.27 is a
lower bound; including subscription amortization would increase it
marginally.

\textbf{Three vendors.} The architectural taxonomy
(monolith/flat/modular) is grounded in three examples. Additional
vendors may reveal architectures that don't fit this taxonomy.

\subsection{Responsible Disclosure}

Two findings have potential user impact:

\textbf{Gemini CLI: save\_memory data loss.} The structural guarantee
that saved preferences are deleted during history compression affects
real users who rely on the memory feature for long sessions. This
finding has been independently confirmed: Google filed and patched a
related symptom (issue \#16213, PR \#16914) without addressing the
schema-level root cause. The schema bug remains in the publicly
available source code as of February 2026.

\textbf{Claude Code: recursive agent spawning.} The Task tool's
\texttt{Tools:\ *} specification includes the Task tool itself, enabling
unbounded recursive agent spawning. This is a known architectural choice
(depth is controlled by other mechanisms) but the system prompt does not
make this explicit.

Both findings derive from publicly available artifacts. We have not
tested whether these structural conditions manifest as runtime failures.
